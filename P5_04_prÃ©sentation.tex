\documentclass[8pt,aspectratio=169,hyperref={unicode=true}]{beamer}

\usefonttheme{serif}
\usepackage{fontspec}
	\setmainfont{TeX Gyre Heros}
\usepackage{unicode-math}
\usepackage{lualatex-math}
	\setmathfont{TeX Gyre Termes Math}
\usepackage{polyglossia}
\setdefaultlanguage[frenchpart=false]{french}
\setotherlanguage{english}
%\usepackage{microtype}
\usepackage[locale = FR,
            separate-uncertainty,
            multi-part-units = single,
            range-units = single]{siunitx}
	\DeclareSIUnit\an{an}
  \DeclareSIUnit{\octet}{o}
\usepackage{amsmath}
\usepackage{amsfonts}
\usepackage{amssymb}
\usepackage{array}
\usepackage{graphicx}
\graphicspath{{./Figures/}}
\usepackage{booktabs}
\usepackage{tabularx}
\usepackage{multirow}
\usepackage{multicol}
    \newcolumntype{L}{>{\raggedright\arraybackslash}X}
    \newcolumntype{R}{>{\raggedleft\arraybackslash}X}
\usepackage{tikz}
\usetikzlibrary{graphs, graphdrawing, arrows.meta} \usegdlibrary{layered, trees}
\usetikzlibrary{overlay-beamer-styles}
\usepackage{subcaption}
\usepackage[]{animate}
\usepackage{float}
\usepackage{csquotes}

\usetheme[secheader
         ]{Boadilla}
\usecolortheme{seagull}
\setbeamertemplate{enumerate items}[default]
\setbeamertemplate{itemize items}{-}
\setbeamertemplate{navigation symbols}{}
\setbeamertemplate{bibliography item}{}
\setbeamerfont{framesubtitle}{size=\large}
\setbeamertemplate{section in toc}[sections numbered]
%\setbeamertemplate{subsection in toc}[subsections numbered]

\title[Segmentez les clients d'un site de e-commerce]
{Projet 5 : Segmentez les clients d'un site de e-commerce}
\author[Lancelot \textsc{Leclercq}]{Lancelot \textsc{Leclercq}} 
\institute[]{}
\date[]{\small{21 janvier 2022}}

\AtBeginSection[]{
  \begin{frame}
  \vfill
  \centering
    \usebeamerfont{title}\insertsectionhead\par%
  \vfill
  \end{frame}
}

\begin{document}
\setbeamercolor{background canvas}{bg=gray!20}
\begin{frame}[plain]
    \titlepage
\end{frame}

\begin{frame}{Sommaire}
    \Large
    \begin{columns}
        \begin{column}{.7\textwidth}
            \tableofcontents[hideallsubsections]
        \end{column}
    \end{columns}
\end{frame}


\section{Introduction}
\subsection{Problématique}
\begin{frame}{\insertsubsection}
    \begin{itemize}
        \item Client : Olist, site de e-commerce
              \begin{itemize}
                  \item Souhaite effectuer une segmentation des clients
                  \item[]
                  \item Comprendre les différents types d'utilisateurs
              \end{itemize}
        \item[]
        \item Objectifs :
              \begin{itemize}
                  \item Fournir une description actionable de la segmentation et de sa logique pour une utilisation optimale
                  \item[]
                  \item Faire une proposition de contrat de maintenance à partir de l'analyse de la stabilité de la segmentation au cours du temps
              \end{itemize}
    \end{itemize}
\end{frame}

\subsection{Jeu de données}
\begin{frame}{\insertsubsection}
    \begin{columns}
        \begin{column}{.4\textwidth}
            \begin{itemize}
                \item Base de données anonymisée du site
                \item[]
                \item Données concernant :
                      \begin{itemize}
                          \item les clients,
                          \item les vendeurs,
                          \item les commandes,
                          \item les produits vendus,
                          \item les commentaires et la satisfaction
                      \end{itemize}
                \item[]
                \item Données des années 2017 et 2018
            \end{itemize}
        \end{column}
        \begin{column}{.6\textwidth}
            \begin{figure}
                \includegraphics[width=\textwidth]{ArchDataOlist.png}
            \end{figure}
        \end{column}
    \end{columns}
\end{frame}

\section{Analyse et transformation des données}
\subsection{structure de la base de données}
\begin{frame}{\insertsection: \insertsubsection}
    \begin{columns}
        \begin{column}{.4\textwidth}
            \begin{itemize}
                \item Fichier central contenant les commandes (olist\_orders\_dataset)
                \item[]
                \item Des identifiants permettent de rassembler les différents fichiers constituant la base de données
                \item[]
                \item Ces données vont nous permettre de calculer des variables plus intéressantes dans le cadre de cette segmentation des clients
            \end{itemize}
        \end{column}
        \begin{column}{.6\textwidth}
            \begin{figure}
                \includegraphics[width=\textwidth]{ArchDataOlist.png}
            \end{figure}
        \end{column}
    \end{columns}
\end{frame}

\subsection{calcul de nouvelles variables}
\begin{frame}{\insertsection: \insertsubsection}
    \begin{itemize}
        \item Études des clients = nécessité de regrouper les données de commandes par client
        \item[]
        \item Permet d'obtenir :
              \begin{itemize}
                  \item la date de la dernière commande
                  \item le prix moyen d'une commande d'un client
                  \item le nombre de commandes par client
                  \item le temps moyen entre deux commandes effectuées par un même client
                  \item la note moyenne données par un client
                  \item le nombre de produits par catégories de produits achetés par un clients
                  \item[] $\vdots$
              \end{itemize}
        \item[]
        \item Utilisation des méthodes traditionnelles de marketing pour classifier les clients : classification RFM
              \begin{itemize}
                  \item Recency : la date à laquelle à été effectuée la dernière commande
                  \item Frequency : le nombre de commande
                  \item Monetary : le prix moyen d'une commandes d'un client
              \end{itemize}
    \end{itemize}
\end{frame}

\subsection{analyse des données}
\begin{frame}{\insertsection: \insertsubsection}
    \begin{columns}
        \begin{column}{.5\textwidth}
            \begin{itemize}
                \item Augmentation du nombre de commandes durant l'années 2017
                \item Développement après le lancement
            \end{itemize}
        \end{column}
        \begin{column}{.5\textwidth}
            \begin{figure}
                \includegraphics[width=\textwidth]{NbCommandesM.pdf}
            \end{figure}
        \end{column}
    \end{columns}
    \begin{columns}
        \begin{column}{.5\textwidth}
            \begin{itemize}
                \item Pic d'achat au 24 novembre 2017 qui correspond au Black Friday
            \end{itemize}
        \end{column}
        \begin{column}{.5\textwidth}
            \begin{figure}
                \includegraphics[width=\textwidth]{NbCommandesJ.pdf}
            \end{figure}
        \end{column}
    \end{columns}
\end{frame}

\begin{frame}{\insertsection: \insertsubsection}
    \begin{columns}[t]
        \begin{column}{.6\textwidth}
            \begin{figure}
                \includegraphics[width=\textwidth]{NbProdCat.pdf}
            \end{figure}
            \begin{itemize}
                \item Très grand nombre de catégories de produits
            \end{itemize}
        \end{column}
        \begin{column}{.4\textwidth}
            \begin{figure}
                \includegraphics[width=\textwidth]{NbProdCat10.pdf}
            \end{figure}
            \begin{itemize}
                \item Regroupement en 10 catégories
            \end{itemize}
        \end{column}
    \end{columns}
\end{frame}

\subsubsection{Visualisation des données de RFM}
\begin{frame}{\insertsection: \insertsubsection}{\insertsubsubsection}
    \begin{columns}[t]
        \begin{column}{.33\textwidth}
            Recency
            \begin{figure}
                \includegraphics[width=\textwidth]{HistRFMlast_purchase_days.pdf}
            \end{figure}
            \begin{itemize}
                \item Beaucoup d'achats entre 0 et 200, mediane autour de 200
            \end{itemize}
        \end{column}
        \begin{column}{.33\textwidth}
            Frequency
            \begin{figure}
                \includegraphics[width=\textwidth]{HistRFMorders_number.pdf}
            \end{figure}
            \begin{itemize}
                \item Majorité des clients ont effectués 2 achats mais quelques exeptions (9, 15)
            \end{itemize}
        \end{column}
        \begin{column}{.33\textwidth}
            Monetary
            \begin{figure}
                \includegraphics[width=\textwidth]{HistRFMmean_payment.pdf}
            \end{figure}
            \begin{itemize}
                \item Dépense médiane autour de 100  certains clients sont très dépensiers jusqu'à 3700
            \end{itemize}
        \end{column}
    \end{columns}
\end{frame}

\section{Essais de différents modèles}
\subsection{KMeans}
\begin{frame}{\insertsection: \insertsubsection}
    \begin{columns}
        \begin{column}{.55\textwidth}
            \begin{figure}
                \includegraphics[width=.49\textwidth]{ScoresKMeans.pdf}
                \includegraphics[width=.49\textwidth]{pieKMeans.pdf}
            \end{figure}
            \begin{itemize}
                \item Les 5 clusters sont clairement interprétables
                      \begin{itemize}
                          \item 0 : clients qui ont fait 3 ou 4 achats d'une valeur moyenne < 450
                          \item 1 : clients qui ont fait 2 achats il y a plus longtemps (environ plus de 250j) d'une valeur moyenne < 600
                          \item 2 : clients qui ont fait 2 achats plutôt recemment (environ moins de 250j) d'une valeur moyenne < 350
                          \item 3 : clients qui ont fait plus de 5 achats d'une valeur moyenne < 400
                          \item 4 : client qui ont fait 2, 3 ou 4 achats de valeurs plus importante (environ > 350)
                      \end{itemize}
            \end{itemize}
        \end{column}
        \begin{column}{.45\textwidth}
            \begin{figure}
                \includegraphics[width=\textwidth]{VisuKMeansClusters.pdf}
            \end{figure}
        \end{column}
    \end{columns}
\end{frame}

\subsection{SpectralClustering}
\begin{frame}{\insertsection: \insertsubsection}
    \begin{columns}
        \begin{column}{.55\textwidth}
            \begin{figure}
                \includegraphics[width=.49\textwidth]{ScoresSpectral.pdf}
                \includegraphics[width=.49\textwidth]{pieSpectral.pdf}
            \end{figure}
            \begin{itemize}
                \item Les 3 clusters sont clairement interprétables
                      \begin{itemize}
                          \item 0 : clients qui ont fait plus de 3 achats d'une valeur moyenne < 700
                          \item 1 : clients qui ont fait 2 achats plutôt recemment (environ moins de 250j) et d'une valeur moyenne < 400
                          \item 2 : clients qui ont fait 2 achats il y a plus longtemps (environ plus de 250j) et/ou d'une valeur moyenne > 400
                      \end{itemize}
            \end{itemize}
        \end{column}
        \begin{column}{.45\textwidth}
            \begin{figure}
                \includegraphics[width=\textwidth]{VisuSpectralClusters.pdf}
            \end{figure}
        \end{column}
    \end{columns}
\end{frame}

\subsection{AgglomerativeClustering}
\begin{frame}{\insertsection: \insertsubsection}
    \begin{columns}
        \begin{column}{.55\textwidth}
            \begin{figure}
                \includegraphics[width=.49\textwidth]{ScoresAgglo.pdf}
                \includegraphics[width=.49\textwidth]{pieAgglo.pdf}
            \end{figure}
            \begin{itemize}
                \footnotesize
                \item Les 6 clusters sont plutôt bien définis mais se chevauches légèrement plus que ceux de l'algorithme KMeans
                      \begin{itemize}
                          \scriptsize
                          \item 0 : clients qui ont fait 2 achats plutôt recemment (environ moins de 250j) d'une valeur moyenne < 500
                          \item 1 : clients qui ont fait plus de 4 achats d'une valeur moyenne < 450
                          \item 2 : clients qui ont fait entre 2 et 4 achats d'une valeur moyenne > 400
                          \item 3 : clients qui ont fait 2 commandes il y a plus longtemps (> 300j pour les montants les moins importants et > 150j pour les montants plus importants)
                          \item 4 : clients qui ont fait 3 achats ou 4 achats de valeurs plus importante (environ > 400) ou il y a plus longtemps (> 250j)
                          \item 5 : clients qui ont fait 2 achats de valeurs très importante (> 3500)
                      \end{itemize}
            \end{itemize}
        \end{column}
        \begin{column}{.45\textwidth}
            \begin{figure}
                \includegraphics[width=\textwidth]{VisuAggloClusters.pdf}
            \end{figure}
        \end{column}
    \end{columns}
\end{frame}

\subsection{DBSCAN}
\begin{frame}{\insertsection: \insertsubsection}
    \begin{columns}
        \begin{column}{.55\textwidth}
            \begin{figure}
                \includegraphics[width=.49\textwidth]{ScoresDBSCAN.pdf}
                \includegraphics[width=.49\textwidth]{pieDBSCAN.pdf}
            \end{figure}
            \begin{itemize}
                \item Cet algorithme cherche lui même le nombre de cluster. Il en a créé 4 et éliminé quelques points. Les quatres clusters correspondent au nombre d'achats effectués
                      \begin{itemize}
                          \item 0 : clients qui ont fait 2 achats
                          \item 1 : clients qui ont fait 3 achats
                          \item 2 : clients qui ont fait 4 achats
                          \item 3 : clients qui ont fait 5 achats
                      \end{itemize}
            \end{itemize}
        \end{column}
        \begin{column}{.45\textwidth}
            \begin{figure}
                \includegraphics[width=\textwidth]{VisuDBSCANClusters.pdf}
            \end{figure}
        \end{column}
    \end{columns}
\end{frame}

\subsection{Birch}
\begin{frame}{\insertsection: \insertsubsection}
    \begin{columns}
        \begin{column}{.55\textwidth}
            \begin{figure}
                \hfill
                \includegraphics[width=.42\textwidth]{ScoresBirch.pdf}
                \hfill
                \includegraphics[width=.42\textwidth]{pieBirch.pdf}
                \hfill
            \end{figure}
            \begin{itemize}
                \footnotesize
                \item Les 9 clusters sont plutôt bien définis mais cela commence à faire un grand nombre de clusters pour différencier des clients
                      \begin{itemize}
                          \scriptsize
                          \item 0 : clients ayant fait entre 6 et 9 commandes
                          \item 1 : clients ayant fait des commandes d'une valeur moyenne plutôt élevée (>800)
                          \item 2 : clients ayant fait 2 ou 3 commandes d'une valeur moyenne plutôt moyenne (entre 400 et 800)
                          \item 3 : clients ayant fait 3 ou 4 commandes d'une valeur moyenne plutôt basse (<400) et il y a plus longtemps (>300j)
                          \item 4 : clients ayant fait 2 ou 3 commandes d'une valeur moyenne plutôt basse (<500) et plus récemment (<300j)
                          \item 5 : clients ayant fait 2 commandes d'une valeur moyenne très élevée (>3500)
                          \item 6 : clients ayant fait 4 ou 5 commandes plutôt récemment (<250j)
                          \item 7 : clients ayant fait un très grand nombre de commandes (15)
                          \item 8 : clients ayant fait 2 commandes d'une valeur moyenne plutôt basse (<450) et il y a plus longtemps (>350j)
                      \end{itemize}
            \end{itemize}
        \end{column}
        \begin{column}{.45\textwidth}
            \begin{figure}
                \includegraphics[width=\textwidth]{VisuBirchClusters.pdf}
            \end{figure}
        \end{column}
    \end{columns}
\end{frame}

\subsection{comparaison des modèles}
\begin{frame}{\insertsection: \insertsubsection}
    \begin{columns}
        \begin{column}{.6\textwidth}
            \begin{itemize}
                \item Plus le coefficient de silhouette est haut plus les clusters sont définis
                \item Il est calculé pour chaque objet et est composé de deux scores :
                      \begin{itemize}
                          \item La distance entre cet objet et les autres objets contenu dans la même classe
                          \item La distance moyenne entre cet objet et ceux contenus dans la classe la plus proche
                      \end{itemize}
                \item[]
                \item Plus le score de Calinski-Harabasz est haut plus les clusters sont denses et bien séparés
                \item Calcul le ratio de la somme des dispersions inter-clusters et de celle des dispersions intra-clusters
                \item[]
                \item Plus le score de Davies-Bouldin est bas plus les clusters sont définis
                \item Calcul la similarité entre les clusters. Cette similarité compare la distance entre les clusters avec la taille des clusters
            \end{itemize}
        \end{column}
        \begin{column}{.4\textwidth}
            \begin{figure}
                \includegraphics[width=\textwidth]{CompareScores.pdf}
            \end{figure}
        \end{column}
    \end{columns}
    \vspace{.5cm}
    Au vu de ces données je choisi d'utiliser l'algorithme KMeans qui a à la fois un coefficient de silhouette et un score de Calinski-Harabasz élevés
\end{frame}

\section{Simulation de l'évolution de la classification}
\subsection{aspect technique}
\begin{frame}{\insertsection: \insertsubsection}
    \begin{itemize}
        \item Utilisation de l'algorithme KMeans à 5 clusters
        \item[]
        \item Utilisation du score ARI : Adjusted Rand Index
        \begin{itemize}
            \item Compare la similarité entre les labels assignés à un même objet pour différentes méthodes de classification
        \end{itemize}
        \item[]
        \item Sélection et ajout (incrémentation) des clients qui nous intéressent (plus de 2 commandes effectuées et reçues) par période donnée (mensuelle, trimestrielle, semestrielle)
        \item[]
        \item Comparaison pour chaque incrémentation avec les données les plus récentes pour les clients correspondants
        \begin{itemize}
            \item L'observation de l'évolution de l'ARI va nous permettre de voir combien de temps les labels assignés restent pertinents
        \end{itemize}
    \end{itemize}
\end{frame}

\subsection{visualisation de l'évolution}
\begin{frame}{\insertsection: \insertsubsection}
    \begin{columns}[t]
        \begin{column}{.5\textwidth}
            \begin{figure}
                \includegraphics[width=\textwidth]{simMAJS.pdf}
            \end{figure}
            \begin{itemize}
                \footnotesize
                \item[\uparrow] Le premier semestre semble être encore assez bien corrélé (ARI>0.9) le second chute assez brutalement (ARI<0.6)
                \item[$\nearrow$] Le premier trimestre reste plutôt bien corrélé (ARI>0.98). La pente augmente ensuite jusqu'au 3è trimestre. On retrouve la chute importante au 4è trimestre
                \item[\rightarrow] Les trois premiers mois les données restent bien corrélées (ARI>0.98) mais après on observe une pente plus importante entre le 3è et le 11è mois avec une chute au 12è mois  
            \end{itemize}
        \end{column}
        \begin{column}{.5\textwidth}
            \begin{figure}
                \includegraphics[width=\textwidth]{simMAJT.pdf}
                \includegraphics[width=\textwidth]{simMAJM.pdf}
            \end{figure}
        \end{column}
    \end{columns}
\end{frame}

\section{Conclusion}
\begin{frame}
    \begin{itemize}
        \item L'algorithme KMeans me parait le plus pertinent car il est le plus performant pour un  nombre de cluster raisonable
        \item Il propose un classification ni trop complexe ni trop simple qui ne permettrait pas de bien comprendre le profil du client
        \item[]
        \item Idéalement renouvellement trimestriel afin de garder des données au plus près de la réalité réalité (ARI>0.98)
        \item[]
        \item Le renouvellement tout les 6 mois me semble rester pertinent car l'ARI reste autour de 0.9
        \item[]
        \item Par contre au delà la chute du score s'accélère et au bout d'un an l'ARI passe en dessous de 0.6
    \end{itemize}
\end{frame}

\end{document}