\documentclass[8pt,aspectratio=169,hyperref={unicode=true}]{beamer}

\usefonttheme{serif}
\usepackage{fontspec}
	\setmainfont{TeX Gyre Heros}
\usepackage{unicode-math}
\usepackage{lualatex-math}
	\setmathfont{TeX Gyre Termes Math}
\usepackage{polyglossia}
\setdefaultlanguage[frenchpart=false]{french}
\setotherlanguage{english}
%\usepackage{microtype}
\usepackage[locale = FR,
            separate-uncertainty,
            multi-part-units = single,
            range-units = single]{siunitx}
	\DeclareSIUnit\an{an}
  \DeclareSIUnit{\octet}{o}
\usepackage{amsmath}
\usepackage{amsfonts}
\usepackage{amssymb}
\usepackage{array}
\usepackage{graphicx}
\graphicspath{{./Figures/}}
\usepackage{booktabs}
\usepackage{tabularx}
\usepackage{multirow}
\usepackage{multicol}
    \newcolumntype{L}{>{\raggedright\arraybackslash}X}
    \newcolumntype{R}{>{\raggedleft\arraybackslash}X}
\usepackage{tikz}
\usetikzlibrary{graphs, graphdrawing, arrows.meta} \usegdlibrary{layered, trees}
\usetikzlibrary{overlay-beamer-styles}
\usepackage{subcaption}
\usepackage[]{animate}
\usepackage{float}
\usepackage{csquotes}

\usetheme[secheader
         ]{Boadilla}
\usecolortheme{seagull}
\setbeamertemplate{enumerate items}[default]
\setbeamertemplate{itemize items}{-}
\setbeamertemplate{navigation symbols}{}
\setbeamertemplate{bibliography item}{}
\setbeamerfont{framesubtitle}{size=\large}
\setbeamertemplate{section in toc}[sections numbered]
%\setbeamertemplate{subsection in toc}[subsections numbered]

\title[Segmentez les clients d'un site de e-commerce]
{Projet 5 : Segmentez les clients d'un site de e-commerce}
\author[Lancelot \textsc{Leclercq}]{Lancelot \textsc{Leclercq}} 
\institute[]{}
\date[]{\small{21 janvier 2022}}

\AtBeginSection[]{
  \begin{frame}
  \vfill
  \centering
    \usebeamerfont{title}\insertsectionhead\par%
  \vfill
  \end{frame}
}

\begin{document}
\setbeamercolor{background canvas}{bg=gray!20}
\begin{frame}[plain]
    \titlepage
\end{frame}

\begin{frame}{Sommaire}
    \Large
    \begin{columns}
        \begin{column}{.7\textwidth}
            \tableofcontents[hideallsubsections]
        \end{column}
    \end{columns}
\end{frame}


\section{Introduction}
\subsection{Problématique}
\begin{frame}{\insertsubsection}
    \begin{itemize}
        \item Client : Olist, site de e-commerce
              \begin{itemize}
                  \item Souhaite effectuer un segmentation des clients
                  \item[]
                  \item Comprendre les différents types d'utilisateurs
              \end{itemize}
        \item[]
        \item Objectifs :
              \begin{itemize}
                  \item Fournir une description actionable de la segmentation et de sa logique pour une utilisation optimale
                  \item[]
                  \item Faire une proposition de contrat de maintenance à partir de l'analyse de la stabilité de la classification au cours du temps
              \end{itemize}
    \end{itemize}
\end{frame}

\subsection{Jeu de données}
\begin{frame}{\insertsubsection}
    \begin{columns}
        \begin{column}{.4\textwidth}
            \begin{itemize}
                \item Base de données anonymisée du site
                \item[]
                \item Données concernant :
                      \begin{itemize}
                          \item les clients,
                          \item les vendeurs,
                          \item les commandes,
                          \item les produits vendus,
                          \item les commentaires et la satisfaction
                      \end{itemize}
                \item[]
                \item Données des années 2017 et 2018
            \end{itemize}
        \end{column}
        \begin{column}{.6\textwidth}
            \begin{figure}
                \includegraphics[width=\textwidth]{ArchDataOlist.png}
            \end{figure}
        \end{column}
    \end{columns}
\end{frame}


\end{document}