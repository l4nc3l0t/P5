\documentclass[8pt,aspectratio=169,hyperref={unicode=true}]{beamer}

\usefonttheme{serif}
\usepackage{fontspec}
	\setmainfont{TeX Gyre Heros}
\usepackage{unicode-math}
\usepackage{lualatex-math}
	\setmathfont{TeX Gyre Termes Math}
\usepackage{polyglossia}
\setdefaultlanguage[frenchpart=false]{french}
\setotherlanguage{english}
%\usepackage{microtype}
\usepackage[locale = FR,
            separate-uncertainty,
            multi-part-units = single,
            range-units = single]{siunitx}
	\DeclareSIUnit\an{an}
  \DeclareSIUnit{\octet}{o}
\usepackage{amsmath}
\usepackage{amsfonts}
\usepackage{amssymb}
\usepackage{array}
\usepackage{graphicx}
\graphicspath{{./Figures/}}
\usepackage{booktabs}
\usepackage{tabularx}
\usepackage{multirow}
\usepackage{multicol}
    \newcolumntype{L}{>{\raggedright\arraybackslash}X}
    \newcolumntype{R}{>{\raggedleft\arraybackslash}X}
\usepackage{tikz}
\usetikzlibrary{graphs, graphdrawing, arrows.meta} \usegdlibrary{layered, trees}
\usetikzlibrary{overlay-beamer-styles}
\usepackage{subcaption}
\usepackage[]{animate}
\usepackage{float}
\usepackage{csquotes}

\usetheme[secheader
         ]{Boadilla}
\usecolortheme{seagull}
\setbeamertemplate{enumerate items}[default]
\setbeamertemplate{itemize items}{-}
\setbeamertemplate{navigation symbols}{}
\setbeamertemplate{bibliography item}{}
\setbeamerfont{framesubtitle}{size=\large}
\setbeamertemplate{section in toc}[sections numbered]
%\setbeamertemplate{subsection in toc}[subsections numbered]

\title[Segmentez les clients d'un site de e-commerce]
{Projet 5 : Segmentez les clients d'un site de e-commerce}
\author[Lancelot \textsc{Leclercq}]{Lancelot \textsc{Leclercq}} 
\institute[]{}
\date[]{\small{21 janvier 2022}}

\AtBeginSection[]{
  \begin{frame}
  \vfill
  \centering
    \usebeamerfont{title}\insertsectionhead\par%
  \vfill
  \end{frame}
}

\begin{document}
\setbeamercolor{background canvas}{bg=gray!20}
\begin{frame}[plain]
    \titlepage
\end{frame}

\begin{frame}{Sommaire}
    \Large
    \begin{columns}
        \begin{column}{.7\textwidth}
            \tableofcontents[hideallsubsections]
        \end{column}
    \end{columns}
\end{frame}


\section{Introduction}
\subsection{Problématique}
\begin{frame}{\insertsubsection}
    \begin{itemize}
        \item Client : Olist, site de e-commerce
              \begin{itemize}
                  \item Souhaite effectuer un segmentation des clients
                  \item[]
                  \item Comprendre les différents types d'utilisateurs
              \end{itemize}
        \item[]
        \item Objectifs :
              \begin{itemize}
                  \item Fournir une description actionable de la segmentation et de sa logique pour une utilisation optimale
                  \item[]
                  \item Faire une proposition de contrat de maintenance à partir de l'analyse de la stabilité de la classification au cours du temps
              \end{itemize}
    \end{itemize}
\end{frame}

\subsection{Jeu de données}
\begin{frame}{\insertsubsection}
    \begin{columns}
        \begin{column}{.4\textwidth}
            \begin{itemize}
                \item Base de données anonymisée du site
                \item[]
                \item Données concernant :
                      \begin{itemize}
                          \item les clients,
                          \item les vendeurs,
                          \item les commandes,
                          \item les produits vendus,
                          \item les commentaires et la satisfaction
                      \end{itemize}
                \item[]
                \item Données des années 2017 et 2018
            \end{itemize}
        \end{column}
        \begin{column}{.6\textwidth}
            \begin{figure}
                \includegraphics[width=\textwidth]{ArchDataOlist.png}
            \end{figure}
        \end{column}
    \end{columns}
\end{frame}

\section{Analyse et transformation des données}
\subsection{structure de la base de données}
\begin{frame}{\insertsection: \insertsubsection}
    \begin{columns}
        \begin{column}{.4\textwidth}
            \begin{itemize}
                \item Fichier central contenant les commandes (olist\_orders\_dataset)
                \item Des identifiants permettent de rassembler les différents fichiers constituant la base de données
                      \begin{itemize}
                          \item order\_id relie les données de paiement, les produits commandés et les données de notation aux commandes,
                          \item[]
                          \item customer\_id relie les identifiants des clients aux commandes,
                          \item[]
                          \item products\_id relie les données concernant les produits aux produits commandés,
                          \item[]
                          \item seller\_id relie les données de vendeurs aux produits commandés,
                          \item[]
                          \item zip\_code\_prefix relie les données de géolocalisation des acheteurs et des vendeurs
                      \end{itemize}
                \item Ces données vont nous permettre de calculer des variables plus intéressantes dans le cadre de cette segmentation des clients
            \end{itemize}
        \end{column}
        \begin{column}{.6\textwidth}
            \begin{figure}
                \includegraphics[width=\textwidth]{ArchDataOlist.png}
            \end{figure}
        \end{column}
    \end{columns}
\end{frame}

\subsubsection{calcul de nouvelles variables}
\begin{frame}{\insertsection: \insertsubsection}
    \begin{itemize}
        \item Études des clients = nécessité de regrouper les données de commandes par client
        \item[]
        \item Permet de compter :
              \begin{itemize}
                  \item la date de la dernière commande
                  \item le prix moyen d'une commande d'un client
                  \item le nombre de commandes par client
                  \item le temps moyen entre deux commandes effectuées par un même client
                  \item la note moyenne données par un client
                  \item le nombre de produits par catégories de produits achetés par un clients
                  \item[] $\vdots$
              \end{itemize}
        \item[]
        \item Utilisation des méthodes traditionnelles de marketing pour classifier les clients : classification RFM
              \begin{itemize}
                  \item Recency : la date à laquelle à été effectuée la dernière commande
                  \item Frequency : le nombre de commande
                  \item Monetary : le prix moyen d'une commandes d'un client
              \end{itemize}
    \end{itemize}
\end{frame}

\end{document}